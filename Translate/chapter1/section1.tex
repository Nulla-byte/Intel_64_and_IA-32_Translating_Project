\section{该手册中包含的英特尔 64 和 IA-32 架构的处理器}
该手册主要是关于最新的英特尔 64 和 IA-32 架构的处理器,包括:
\begin{itemize}
	\setlength\itemsep{-0.3em}
	\item 奔腾处理器
	\item P6 家族处理器
	\item 奔腾 4 处理器
	\item 奔腾 M 处理器
	\item 英特尔至强处理器
	\item 奔腾 D 处理器
	\item 奔腾处理器至尊版
	\item 64 位英特尔至强处理器
	\item 英特尔酷睿双核处理器
	\item 英特尔酷睿单核处理器
	\item 双核英特尔至强处理器 LV
	\item 英特尔酷睿四核处理器 Q6000 系列
	\item 英特尔至强处理器 3000, 3200 系列
	\item 英特尔至强处理器 5000 系列
	\item 英特尔至强处理器 5100, 5300 系列
	\item 英特尔酷睿至尊处理器 X7000 和 X6800 系列
	\item 英特尔酷睿至尊处理器 QX6000 系列
	\item 英特尔至强处理器 7100 系列
	\item 英特尔奔腾双核处理器
	\item 英特尔至强 7200, 7300 系列
	\item 英特尔 至强 5200, 5400, 7400 系列
	\item 英特尔 酷睿2 至尊处理器 QX9000 和 X9000 系列
	\item 英特尔 酷睿2 四核处理器 Q9000 系列
	\item 英特尔 酷睿2 双核处理器 E8000, T9000 系列
	\item 英特尔 凌动 处理器家族
	\item 英特尔 凌动 处理器 200, 300, D400, D500, D2000, N200, N400, N2000, E2000, Z500, Z60, Z2000, C1000 系列来自于 45 nm 和 32 nm 工艺
	\item 英特尔 酷睿 i7 处理器
	\item 英特尔 酷睿 i5 处理器
	\item 英特尔 至强 处理器 E7-8800/4800/2800 产品家族
	\item 英特尔 酷睿 i7-3930K 处理器
	\item 二代英特尔 酷睿 i7-2xxx,英特尔 酷睿 i5-2xxx,英特尔 酷睿 i3-2xxx 处理器系列
	\item 英特尔 至强 处理器 E3-1200 产品家族
	\item 英特尔 至强 处理器 E5-2400/1400 产品家族
	\item 英特尔 至强 处理器 E5-4600/2600/1600 产品家族
	\item 三代英特尔 酷睿处理器
	\item 英特尔 至强 处理器 E3-1200 v2 产品家族
	\item 英特尔 至强 处理器 E5-2400/1400 v2 产品家族
	\item 英特尔 至强 处理器 E5-4600/2600/1600 v2 产品家族
	\item 英特尔 至强 处理器 E7-8800/4800/2800 v2 产品家族
	\item 四代英特尔 酷睿处理器
	\item 英特尔 酷睿 M 处理器家族
	\item 英特尔 酷睿 i7-59xx 处理器至尊版
	\item 英特尔 酷睿 i7-49xx 处理器至尊版
	\item 英特尔 至强处理器 E3-1200 v3 产品家族
	\item 英特尔 至强处理器 E5-2600/1600 v3 产品家族
	\item 五代英特尔 酷睿处理器
	\item 英特尔 至强处理器 D-1500 产品家族
	\item 英特尔 至强处理器 E5 v4 家族
	\item 英特尔 凌动处理器 X7-Z8000 和 X5-Z8000 系列
	\item 英特尔 凌动处理器 Z3400 系列
	\item 英特尔 凌动处理器 Z3500 系列
	\item 六代英特尔 酷睿处理器
	\item 英特尔 至强处理器 E3-1500m v5 产品家族
	\item 七代英特尔 酷睿处理器
	\item 英特尔 至强 融核 处理器 3200, 5200, 7200 系列
	\item 英特尔 至强处理器可扩展系列
	\item 八代英特尔 酷睿处理器
	\item 英特尔 至强 Phi 处理器 7215, 7285, 7295 系列
\end{itemize}

\newpage

P6 系列处理器是基于 P6 系列架构的 IA-32 处理器。这包括奔腾 Pro,奔腾 II,奔腾 III 和 奔腾 III 至强 处理器。

奔腾 4,奔腾 D 和奔腾处理器至尊版基于英特尔 NetBurst 微体系结构。 大多数早期的英特尔 至强 处理器都基于英特尔 NetBurst 微体系结构。 英特尔至强处理器 5000, 7100 系列基于英特尔 NetBurst 微体系结构。

英特尔酷睿双核,英特尔酷睿单核以及双核英特尔至强处理器 LV 是基于改进的奔腾 M 处理器微体系结构。

英特尔至强处理器 3000, 3200, 5100, 5300, 7200 和 7300 系列,英特尔奔腾双核,英特尔酷睿 2 双核,英特尔酷睿 2 四核和英特尔酷睿 2 至尊处理器基于英特尔酷睿微体系结构。

英特尔至强处理器 5200,5400,7400 系列,英特尔酷睿 2 四核处理器 Q9000 系列和英特尔酷睿 2 至尊处理器 QX9000,X9000 系列,英特尔酷睿 2 处理器 E8000 系列均基于增强型英特尔酷睿微体系结构。

英特尔凌动处理器 200, 300, D400, D500, D2000, N200, N400, N2000, E2000, Z500, Z600, Z2000, C1000 系列基于英特尔凌动微体系结构,支持英特尔 64 架构。

P6 家族,奔腾 M,英特尔酷睿单核,英特尔酷睿双核处理器,双核英特尔至强处理器 LV 以及早期的奔腾 4 和英特尔至强处理器均支持 IA-32 架构。英特尔凌动处理器 Z5xx 系列支持 IA-32 架构。

英特尔至强处理器 3000, 3200, 5000, 5100, 5200, 5300, 5400, 7100, 7200, 7300, 7400 系列,英特尔酷睿 2 双核处理器,英特尔酷睿 2 至尊,英特尔酷睿 2 四核处理器 ,奔腾 D 处理器,奔腾双核处理器,新一代奔腾 4 和英特尔至强处理器系列均支持英特尔 64 架构。

英特尔酷睿 i7 处理器和英特尔至强处理器 3400, 5500, 7500 系列基于代号 Nehalem 的 45 纳米英特尔微体系结构。Westmere 是 32 纳米版本的 Nehalem。英特尔至强处理器 5600 系列,英特尔至强处理器 E7 和各种英特尔酷睿 i7, i5, i3 处理器均基于代号 Westmere 的英特尔微体系结构。 这些处理器支持英特尔 64 架构。

英特尔至强处理器 E5 系列,英特尔至强处理器 E3-1200 系列,英特尔至强处理器 E7-8800/4800/2800 产品系列,英特尔酷睿 i7-3930K 处理器和第二代英特尔酷睿 i7-2xxx,英特尔酷睿 i5-2xxx,英特尔酷睿 i3-2xxx 处理器系列基于代号 Sandy Bridge 的英特尔微体系结构 ,支持英特尔 64 架构。

英特尔至强处理器 E7-8800/4800/2800 v2 产品系列,英特尔至强处理器 E3-1200 v2 产品系列和第三代英特尔酷睿处理器基于代号 Ivy Bridge 的英特尔微体系结构并支持英特尔 64 架构。

英特尔至强处理器 E5-4600/2600/1600 v2 产品系列,英特尔至强处理器 E5-2400/1400 v2 产品系列和英特尔酷睿 i7-49xx 处理器至尊版基于英特尔微体系结构代码名称 Ivy Bridge-E 并支持英特尔 64 架构。

英特尔至强处理器 E3-1200 v3 产品系列和第四代英特尔酷睿处理器基于代号 Haswell 的英特尔微体系结构,并支持英特尔 64 架构。

英特尔至强处理器 E5-2600/1600 v3 产品系列和英特尔酷睿 i7-59xx 处理器至尊版基于英特尔微体系结构代号 Haswell-E,并支持英特尔 64 架构。

英特尔凌动处理器 Z8000 系列基于代号 Airmont 的英特尔微体系结构。

英特尔凌动处理器 Z3400 系列和英特尔凌动处理器 Z3500 系列基于代号 Silvermont 的英特尔微体系结构。

英特尔酷睿 M 处理器系列,第 5 代英特尔酷睿处理器,英特尔至强处理器 D-1500 产品系列和英特尔至强处理器 E5 v4 系列均基于代号 Broadwell 的英特尔微体系结构,并支持英特尔 64 架构。

英特尔至强处理器可扩展系列,英特尔至强处理器 E3-1500m v5 产品系列和第六代英特尔酷睿处理器基于代号 Skylake 的英特尔微体系结构,并支持英特尔 64 架构。

第 7 代英特尔酷睿处理器基于代号 Kaby Lake 的英特尔微体系结构,支持英特尔 64 架构。

至强融核处理器 3200, 5200, 7200 系列基于代码 Knights Landing 的英特尔微体系结构,支持 64 架构。

第 8 代英特尔酷睿处理器基于代号 Coffee Lake 的英特尔微体系结构并支持英特尔 64 架构。

至强融核处理器 7215, 7285, 7295 系列基于代号 Knights Mill 的英特尔微体系结构,支持英特尔 64 架构。

IA-32 架构是英特尔 32 位微处理器的指令集架构和编程环境。英特尔 64 架构是指令集架构和编程环境,是英特尔 32 位和 64 位架构的超集。它与 IA-32 架构兼容。
\newpage
\newgeometry{left=0.9cm,right=0.9cm,top=2cm,bottom=2cm}
\begin{multicols}{2}
\begin{table}[H]
  \centering
  \caption{按产品家族分类}
  \scriptsize
  \renewcommand\arraystretch{1.011}
    \begin{tabular}{|c|l|l|}
    \hline
    \textbf{家族} & \multicolumn{1}{c|}{\textbf{处理器名称/系列}} & \multicolumn{1}{c|}{\textbf{微体系结构}} \\
    \hline
    \multirow{3}[12]{*}{奔腾家族} & 奔腾 Pro, 奔腾 II, 奔腾 III, 奔腾 III 至强 & P6 系列微体系结构 \\
\cline{2-3}          & 4     & NetBurst \\
\cline{2-3}          & M     & 奔腾 M 微体系结构 \\
\cline{2-3}          & D     & NetBurst \\
\cline{2-3}          & 至尊版   & NetBurst \\
\cline{2-3}          & 双核    & 酷睿微体系结构 \\
    \hline
    \multirow{12}[46]{*}{至强家族} & 双核 LV & 改进的奔腾 M 微体系结构 \\
\cline{2-3}          & 3000, 3200 系列 & 酷睿微体系结构 \\
\cline{2-3}          & 5000 系列 & NetBurst \\
\cline{2-3}          & 5100, 5300 系列 & 酷睿微体系结构 \\
\cline{2-3}          & 7100 系列 & NetBurst \\
\cline{2-3}          & 7200, 7300 系列 & 酷睿微体系结构 \\
\cline{2-3}          & 5200, 5400, 7400 系列 & 增强型酷睿微体系结构 \\
\cline{2-3}          & E7-8800/4800/2800 产品家族 & Sandy Bridge \\
\cline{2-3}          & E3-1200 产品家族 & Sandy Bridge \\
\cline{2-3}          & E5-2400/1400 产品家族 & Sandy Bridge \\
\cline{2-3}          & E5-4600/2600/1600 产品家族 & Sandy Bridge \\
\cline{2-3}          & E3-1200 v2 产品家族 & Ivy Bridge \\
\cline{2-3}          & E5-2400/1400 v2 产品家族 & Ivy Bridge-E \\
\cline{2-3}          & E5-4600/2600/1600 v2 产品家族 & Ivy Bridge-E \\
\cline{2-3}          & E7-8800/4800/2800 v2 产品家族 & Ivy Bridge \\
\cline{2-3}          & E3-1200 v3 产品家族 & Haswell \\
\cline{2-3}          & E5-2600/1600 v3 产品家族 & Haswell-E \\
\cline{2-3}          & D-1500 产品家族 & Broadwell \\
\cline{2-3}          & E5 v4 家族 & Broadwell \\
\cline{2-3}          & E3-1500m v5 产品家族 & Skylake \\
\cline{2-3}          & 融核 3200, 5200, 7200 系列 & Knights Landing \\
\cline{2-3}          & 可扩展系列 & Skylake \\
\cline{2-3}          & 融核 7215, 7285, 7295 系列 & Knights Mill \\
    \hline
    \multirow{12}[42]{*}{酷睿家族} & 双核    & 改进的奔腾 M 微体系结构 \\
\cline{2-3}          & 单核    & 改进的奔腾 M 微体系结构 \\
\cline{2-3}          & 酷睿2 双核 & 酷睿微体系结构 \\
\cline{2-3}          & 酷睿2 四核 Q6000 系列 & 酷睿微体系结构 \\
\cline{2-3}          & 酷睿2 至尊 X7000 , X6800 系列 & 酷睿微体系结构 \\
\cline{2-3}          & 酷睿2 至尊 QX6000 系列 & 酷睿微体系结构 \\
\cline{2-3}          & 酷睿2 至尊 QX9000, X9000 系列 & 增强型酷睿微体系结构 \\
\cline{2-3}          & 酷睿2 四核 Q9000 系列 & 增强型酷睿微体系结构 \\
\cline{2-3}          & 酷睿2 双核 E8000, T9000 系列 & 增强型酷睿微体系结构 \\
\cline{2-3}          & i7, i5 & Westmere \\
\cline{2-3}          & i7-3930K & Sandy Bridge \\
\cline{2-3}          & 二代 i7-2xxx, i5-2xxx, i3-2xxx 系列 & Sandy Bridge \\
\cline{2-3}          & 三代 i7-3xxx, i5-3xxx, i3-3xxx 系列 & Ivy Bridge \\
\cline{2-3}          & 四代 i7-4xxx, i5-4xxx, i3-4xxx 系列 & Haswell \\
\cline{2-3}          & M     & Broadwell \\
\cline{2-3}          & i7-59xx 至尊版 & Haswell-E \\
\cline{2-3}          & i7-49xx 至尊版 & Ivy Bridge-E \\
\cline{2-3}          & 五代 i7-5xxx, i5-5xxx, i3-5xxx 系列 & Broadwell \\
\cline{2-3}          & 六代 i7-6xxx, i5-6xxx, i3-6xxx 系列 & Skylake \\
\cline{2-3}          & 七代 i7-7xxx, i5-7xxx, i3-7xxx 系列 & Kaby Lake \\
\cline{2-3}          & 八代 i7-8xxx, i5-8xxx, i3-8xxx 系列 & Coffee Lake \\
    \hline
    \multirow{4}[8]{*}{凌动家族} & \multicolumn{1}{p{16.5em}|}{200, 300, D400, D500, D2000,\newline{}N200, N400, N2000,\newline{}E2000, Z500, Z60, Z2000, C1000 系列} & 凌动微体系结构 \\
\cline{2-3}          & X7-Z8000 和 X5-Z8000 系列 & Airmont \\
\cline{2-3}          & Z3400 系列 & Silvermont \\
\cline{2-3}          & Z3500 系列 & Silvermont \\
    \hline
    \end{tabular}
\end{table}
\begin{table}[H]
  \centering
  \caption{按微体系结构分类}
  \scriptsize
    \begin{tabular}{|l|l|}
    \hline
    \multicolumn{1}{|c|}{\textbf{微体系结构}} & \multicolumn{1}{c|}{\textbf{处理器名称/系列}} \\
    \hline
    P6 系列微体系结构 & 奔腾 Pro, 奔腾 II, 奔腾 III, 奔腾 III 至强 \\
    \hline
    \multirow{2}[10]{*}{NetBurst} & 奔腾 4 \\
\cline{2-2}          & 奔腾 D \\
\cline{2-2}          & 奔腾至尊版 \\
\cline{2-2}          & 至强 5000 系列 \\
\cline{2-2}          & 至强 7100 系列 \\
    \hline
    奔腾 M 微体系结构 & 奔腾 M \\
    \hline
    \multirow{1}[6]{*}{改进的奔腾 M 微体系结构} & 至强双核 LV \\
\cline{2-2}          & 酷睿双核 \\
\cline{2-2}          & 酷睿单核 \\
    \hline
    \multirow{5}[16]{*}{酷睿微体系结构} & 奔腾双核 \\
\cline{2-2}          & 至强 3000, 3200 系列 \\
\cline{2-2}          & 至强 5100, 5300 系列 \\
\cline{2-2}          & 至强 7200, 7300 系列 \\
\cline{2-2}          & 酷睿2 双核 \\
\cline{2-2}          & 酷睿2 四核 Q6000 系列 \\
\cline{2-2}          & 酷睿2 至尊 X7000 , X6800 系列 \\
\cline{2-2}          & 酷睿2 至尊 QX6000 系列 \\
    \hline
    \multirow{2}[8]{*}{增强型酷睿微体系结构} & 至强 5200, 5400, 7400 系列 \\
\cline{2-2}          & 酷睿2 至尊 QX9000, X9000 系列 \\
\cline{2-2}          & 酷睿2 四核 Q9000 系列 \\
\cline{2-2}          & 酷睿2 双核 E8000, T9000 系列 \\
    \hline
    凌动微体系结构 & \multicolumn{1}{p{17.39em}|}{凌动 200, 300, D400, D500, D2000,\newline{}N200, N400, N2000,\newline{}E2000, Z500, Z60, Z2000, C1000 系列} \\
    \hline
    Westmere & 酷睿 i7, i5 \\
    \hline
    \multirow{3}[12]{*}{Sandy Bridge} & 至强 E7-8800/4800/2800 产品家族 \\
\cline{2-2}          & 至强 E3-1200 产品家族 \\
\cline{2-2}          & 至强 E5-2400/1400 产品家族 \\
\cline{2-2}          & 至强 E5-4600/2600/1600 产品家族 \\
\cline{2-2}          & 酷睿 i7-3930K \\
\cline{2-2}          & 酷睿二代 i7-2xxx, i5-2xxx, i3-2xxx 系列 \\
    \hline
    \multirow{2}[6]{*}{Ivy Bridge} & 至强 E3-1200 v2 产品家族 \\
\cline{2-2}          & 至强 E7-8800/4800/2800 v2 产品家族 \\
\cline{2-2}          & 酷睿三代 i7-3xxx, i5-3xxx, i3-3xxx 系列 \\
    \hline
    \multirow{2}[6]{*}{Ivy Bridge-E} & 至强 E5-2400/1400 v2 产品家族 \\
\cline{2-2}          & 至强 E5-4600/2600/1600 v2 产品家族 \\
\cline{2-2}          & 酷睿 i7-49xx 至尊版 \\
    \hline
    \multirow{1}[4]{*}{Haswell} & 至强 E3-1200 v3 产品家族 \\
\cline{2-2}          & 酷睿四代 i7-4xxx, i5-4xxx, i3-4xxx 系列 \\
    \hline
    \multirow{1}[4]{*}{Haswell-E} & 至强 E5-2600/1600 v3 产品家族 \\
\cline{2-2}          & 酷睿 i7-59xx 至尊版 \\
    \hline
    Airmont & 凌动 X7-Z8000 和 X5-Z8000 系列 \\
    \hline
    \multirow{1}[4]{*}{Silvermont} & 凌动 Z3400 系列 \\
\cline{2-2}          & 凌动 Z3500 系列 \\
    \hline
    \multirow{2}[8]{*}{Broadwell} & 至强 D-1500 产品家族 \\
\cline{2-2}          & 至强 E5 v4 家族 \\
\cline{2-2}          & 酷睿 M \\
\cline{2-2}          & 酷睿五代 i7-5xxx, i5-5xxx, i3-5xxx 系列 \\
    \hline
    \multirow{1}[6]{*}{Skylake} & 至强 E3-1500m v5 产品家族 \\
\cline{2-2}          & 至强可扩展系列 \\
\cline{2-2}          & 酷睿六代 i7-6xxx, i5-6xxx, i3-6xxx 系列 \\
    \hline
    Kaby Lake & 酷睿七代 i7-7xxx, i5-7xxx, i3-7xxx 系列 \\
    \hline
    Knights Landing & 至强融核 3200, 5200, 7200 系列 \\
    \hline
    Coffee Lake & 酷睿八代 i7-8xxx, i5-8xxx, i3-8xxx 系列 \\
    \hline
    Knights Mill & 至强融核 7215, 7285, 7295 系列 \\
    \hline
    \end{tabular}
\end{table}
\end{multicols}
\restoregeometry